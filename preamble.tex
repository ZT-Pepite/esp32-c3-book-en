% Put this file to the same directory of the Chapter *.tex file.
% !! Also, change the definition of chapter name to normal ones,
% e.g., change "\chapter[Firmware Upgrade and Version Management]{\chaptertitle{Firmware Upgrade and Version Management}{Firmware Upgrade and\newline Version Management}}"
% to just "\chapter{Firmware Upgrade and Version Management}"
% Delete the two lines about "fancyfoot" and the {\makeatletter...} (if any) before the chapter title.

\usepackage{graphicx}
\usepackage[export]{adjustbox}
\usepackage{subcaption}
\graphicspath{{Pics/}}
\usepackage{charter}
\usepackage{fancyhdr}
\usepackage{enumerate}
\usepackage{enumitem}
\usepackage{siunitx}
\usepackage{amssymb}
\usepackage{listings}
\usepackage{longtable}
\usepackage{multicol}
\usepackage{multirow}
\usepackage{hhline}
\usepackage{parskip}
\setlength{\parskip}{6pt}
\usepackage{ragged2e}
\usepackage{geometry} %margin
\geometry{left=2.1cm,right=2.1cm,top=3cm,bottom=3cm}
\usepackage{setspace}
\SetSinglespace{1.2}
\singlespacing
\renewcommand{\ttfamily}{\fontfamily{pcr}\selectfont}

\usepackage[,table]{xcolor}
\definecolor{LightBlue}{cmyk}{0.16,0.03,0.04,0}
\definecolor{Title}{cmyk}{0.8,0.1,0,0.3}
\setlength{\arrayrulewidth}{0.3mm}
\setlength{\tabcolsep}{2pt}
\setlength{\headheight}{15pt}
\renewcommand{\arraystretch}{1}
\usepackage{tikz}
\usepackage{nicematrix}
\newcommand*\circled[1]{\tikz[baseline=(char.base)]{\node[shape=circle,draw,inner sep=1pt] (char) {#1};}}

\newcolumntype{s}{>{\columncolor{Title}\RaggedLeft} m{3.5em}} % columntype for chapter title
\newcolumntype{d}{>{\columncolor{LightBlue}\RaggedRight} m{\textwidth}} % columntype for code bloc
\newcolumntype{a}{>{\columncolor{LightBlue}\RaggedRight} m{0.96\textwidth}} % columntype for secondary code bloc

\usepackage{titlesec}
\usepackage[hidelinks]{hyperref}
\urlstyle{same}

\newenvironment{term}[1]
{\subsubsection{#1}}
{}

\newenvironment{secterm}[1]
{\subsubsection{#1}}
{}

\newenvironment{codebloc}
{\quote}
{\endquote}

\newcommand{\note}[2][NOTE]{ %Note/Tips
\begin{quote}
    \textbf{#1}

    #2
\end{quote}
}

\newcommand{\secnote}[2][NOTE]{ %Note/Tips
\begin{quote}
    \textbf{#1}

    #2
\end{quote}
}

\renewcommand{\texttt}[1]{
    !!code here!! #1
}